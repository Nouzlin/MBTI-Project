\section{Theory} \label{sec:theory}

This section will briefly cover the basics of Myers-Briggs Type Indicator, language modelling, feature representations, common supervised learning models and evaluation metrics.

\subsection{Myers-Briggs Type Indicator}
The foundation of the Myers-Briggs Type Indicator (MBTI) is derived from the work of Carl G. Jung \cite{jung2014psychological}.
The idea behind MBTI was to make the concepts observed by Jung more accessible to the public \cite{mbti}.
Table \ref{tab:mbti-types} contains the 16 personality types defined by Myers-Briggs \cite{mbti}.

\subsubsection{Introversion and Extroversion}
The introversion (I) and extroversion (E) pair is used to describe what an individual focuses on and draws their energy from.
Introversion means that the individual focuses on his/her own "world", which means that they draw energy from ideas and memories and from self-reflection \cite{mbti-IE}.
Extroversion is the opposite side; energy is drawn from activities and events with others \cite{mbti-IE}.
Problems for people with the extroversion type are typically solved by talking and discussing with others out loud \cite{mbti-IE}.

\subsubsection{Intuition and Sensing}
The intuition (N) and sensing (S) pair describes how individuals handle information.
Intuition means that individuals pay more attention to patterns, impressions, theoretical concepts and abstract theories \cite{mbti-NS}.
Sensing is, on the other hand, connected to the five senses.
Information is gathered from the physical connection to the world, e.g., what an individual hear, smell and touch \cite{mbti-NS}.

\subsubsection{Thinking and Feeling}
The thinking (T) and feeling (F) pair covers how individuals perform decision making.
Individuals with the thinking type typically analyse ground truths and use logical reasoning to make a decision; pros and cons are analysed \cite{mbti-TF}.
The feeling type describes people who take the opinion of others into consideration \cite{mbti-TF}.
They weigh the outcome of the situation and try to maintain harmony, by caring about the point-of-view of others \cite{mbti-TF}.

\subsubsection{Judging and Perceiving}
Judging (J) and perceiving (P) pair is the final pair of the MBTI.
The pair is related to how others typically see the individual.
People with the judging trait usually live more structured lives than those with the perceiving trait \cite{mbti-JP}.
The major difference between this pair and the pairs covered before, is that this trait is based on the side(s) the individual let others see \cite{mbti-JP}.
For example, an individual can be structured and live ordered lives when alone, but in the interaction with others they tend to be interpreted as spontaneous and open-ended \cite{mbti-JP}.

\begin{table}[!t]
    %% increase table row spacing, adjust to taste
    \renewcommand{\arraystretch}{3}
    \caption{Myers-Briggs Type Indicator Table}
    \label{tab:mbti-types}
    \centering
    %% Some packages, such as MDW tools, offer better commands for making tables
    %% than the plain LaTeX2e tabular which is used here.
    \begin{tabular}{|c|c|c|c|}
        \hline
        ISTJ & ISFJ & INFJ & INTJ \\
        \hline
        ISTP & ISFP & INFP & INTP \\
        \hline
        ESTP & ESFP & ENFP & ENTP \\
        \hline
        ESTJ & ESFJ & ENFJ & ENTJ \\
        \hline
    \end{tabular}
\end{table}

\subsection{Document Representation}
A document $d_i$ can be mathematically represented as a vector $d_i = (w_1, \ldots, w_N)$, where $w_i$ is the $i$-th word (term) in the document.
The words (terms) could, for example, be unigrams (Section \ref{sec:theory-bow}), bigrams (Section \ref{sec:theory-bigram}) or letters (Section \ref{sec:theory-letters}).

\subsection{Latent Dirichlet Allocation}
The latent Dirichlet allocation (LDA) model produces top topic terms for each topic in the model.
It can also be used to calculate the topic distribution for a document.

The LDA model, described by Blei et al. in \cite{Blei2003}, is a probabilistic generative model built on the notion of underlying topics.
A document can belong to several topics simultaneously (with a probability $p(\alpha_i)$ of belonging to topic $\alpha_i$).
All the terms in a collection can belong to their own topic(s).
Each term in a document is assumed to be generated from a topic (following a multinomial distribution with a latent parameter drawn from a Dirichlet distribution).
The LDA model can thus, for a given document, calculate the probability of that document belonging to some topic $\alpha_i$.

\subsection{Features} \label{sec:theory-features}

\subsubsection{Normalisation}
Normalisation transforms the data values above $0$ in the feature vector to be in the range $(0, 1]$.

\subsubsection{Bag-Of-Words} \label{sec:theory-bow}
The bag-of-words model is a simple representation of a document.
The model ignores the order of words and the words occurring before the $i$-th word ($w_i$), see Equation \ref{eq:bow}.

\begin{equation} \label{eq:bow}
    p(w_i|w_1, \ldots, w_{i-1}) = p(w_i)
\end{equation}

The feature vector consists of occurrence counts of words in a dictionary, where each word is represented by an index in the vector.
The dictionary can be either predefined or created from the training set.
One limitation of the bag-of-words model is how to handle words not present in the training corpus.
Unknown words can be either discarded or replaced with an \texttt{<UNK>} tag.
The count vector can then be normalised, smoothed or used as-is.

\subsubsection{TF-IDF}
Term frequency and inverted document frequency (TF-IDF) can be constructed from a bag-of-words model.
The TF is the word (term) counts for a document (typically normalised) and the IDF is the document frequency for that word

\begin{equation*} \label{eq:idf}
    IDF_{k_i} = log(N / n_i)
\end{equation*}
where $N$ is the number of documents and $n_i$ is the number of documents where term $k_i$ appears.

\subsubsection{Bigram Model} \label{sec:theory-bigram}
The bigram model is similar to the bag-of-words model in Section \ref{sec:theory-bow}.
Instead of using counts of single words, the bigram model uses pairs of adjacent words as features, see Equation \ref{eq:bigram}.

\begin{equation} \label{eq:bigram}
    p(w_i|w_1, \ldots, w_{i-1}) = p(w_i|w_{i-1})
\end{equation}

The appearance order for different pairs is ignored after construction; it is only the counts of pairs that are interesting.

\subsubsection{Bag-Of-Letters} \label{sec:theory-letters}
The bag-of-letters feature vector is a simplified variant of the bag-of-words model.
The features are constructed from a predefined alphabet, consisting of letters and characters.
Similar to the bag-of-words model, the symbols in the alphabet are counted for each document.
The feature vector can then be normalised or used as-is.

\subsubsection{Topic Distribution (TD)}
The LDA model can calculate the probability distribution $$T_{d_i}=[p(\alpha_1),\ldots,p(\alpha_t)]$$ over all $t$ topics ($\alpha_1,\ldots,\alpha_t$) for a given document $d_i$.
$p(\alpha_i)$ is the probability that the document $d_i$ belongs to the topic $\alpha_i$.
The distribution can thus be used to determine which topic(s) the document was generated from.
This probability distribution can be directly used as a feature vector for various models.

\subsubsection{Topic Terms (TT)} \label{sec:theory-topic-terms}
In \cite{Chen2016}, Chen et al. try to use LDA to improve the performance of Support Vector Machines for text classification.
In their work they combine the topic distribution of a document from a LDA model together with the top terms for the topics.
The same approach is tested in this work.
The topic terms feature vectors are constructed by first calculating the topic distribution $T_{d_i}$ for each document $d_i$.
Then, for each topic $\alpha_i$, the $n$ top terms ($\beta_{i, 1}, \ldots, \beta_{i, n}$) are used to construct the feature vector.
The topic distribution and the topic terms can be retrieved from a LDA model with $t$ topics.
The feature vector $f_{d_i}$ for a document $d_i$  is thus defined by 
\begin{equation*}
    f_{d_i}=[\beta_1, \ldots, \beta_t]
\end{equation*}
where $\beta_i$ is short-hand notation for the values $\beta_{i, 1}, \ldots, \beta_{i, n}$. 
The dimension of a topic terms feature vector for one document is $t*n$ feature values.
Each value in the feature vector is initialised to zero.

The topic distribution determines how many top terms shall be counted from each topic in the LDA model.
If topic $\alpha_i$  has probability $p_i$, then the number of top terms to be used ($n_i$) from topic $\alpha_i$ is given by Equation \ref{eq:n-top-topic-terms}.

\begin{equation} \label{eq:n-top-topic-terms}
    n_i=Round(p_i*n)
\end{equation}

The $n-n_i$ top terms not used in topic $\alpha_i$ have their feature values set to zero.
The feature value for each of the $n_i$ top terms is set according to Equation \ref{eq:feature-value}.

\begin{equation} \label{eq:feature-value}
    feature\ value =\left\{
                \begin{array}{ll}
                  1+TF, \quad \ if\ term\ present\\
                  1, \qquad\qquad otherwise
                \end{array}
              \right.
\end{equation}

If the term is present in document $d_i$, it is counted and the feature value is set to $1+TF$.
$TF$  is the number of times the term occurs in document $d_i$.  
If the term is not in the document, the feature value for the term is set to $1$.
If a document $d_i$  is calculated by the LDA to belong to a topic $\alpha_i$, then Equation \ref{eq:feature-value} would give more weight to the top terms that do occur in $d_i$, compared to the missing ones.

For example, a LDA with five topics ($t=5$) and a document $d_i$  with the topic distribution 
\[T_{d_i}=[0.01, 0.15, 0.75, 0.09, 0]\]
and $n=10$ top topic terms would result in a feature vector $f_{d_i}$ where:

\begin{itemize}
    \item $n_1$ is rounded down to zero, thus setting all values for the 10 top topic terms in $\alpha_1$ to zero.
    \item $n_2$ is rounded up to two, which means that only the values for the two top terms in $\alpha_2$ are updated to $1+TF$.
    \item $n_3$  is rounded up to 8, which means the values for the 8 top terms in topic $\alpha_3$ are updated.
    \item $n_4$ is rounded up to one, which means that only the value for the top term in topic $\alpha_4$ is updated.
    \item $n_5$  is zero, thus all 10 values are set to zero (same as $n_1$).
\end{itemize}

\subsection{Models} \label{sec:theory-models}

Various supervised learning models can be used for classification.
This section will very briefly cover the models used in this work.

\subsubsection{Linear Support Vector Classification}
The Linear Support Vector Classifier (LinSVC) can efficiently handle sparse features\footnote{http://scikit-learn.org/stable/modules/generated/sklearn.svm.LinearSVC.html}.
The SVC has a slightly different mathematical formulation than the traditional Support Vector Machine (SVM), but the idea behind the model is the same as in the SVM case.
A hyperplane is constructed in order to try and separate the different classes with as big of a margin as possible.
The margin is defined as the distance from the hyperplane to the nearest point.
Slack can be introduced in order to allow for misclassifications, as needed in the case of data not being linearly separable.
Regularisation is typically used in order to control the slack, otherwise the margin would be maximised by potentially misclassifying all data points.

\subsubsection{Logistic Regression Classifier}
The Logistic Regression (LR) classifier, also called logit or MaxEnt, is a probabilistic classifier consisting of linear combinations of predictor functions and weights.
The predictor function calculates the probability of a data point $X_i$ belonging to a class $K_i$ ($p(K_i|X_i)$), divided by the probability that it belongs to all other classes ($p(\{K \setminus K_i\} | X_i)$).

\subsubsection{Extra Trees Classifier}
The Extra Trees (ET) classifier is a variant of the well-known Random Forest (RF) classifier.
Instead of choosing the best decision boundary at each split it continuously chooses attributes and cut-points at random \cite{Geurts2006}.
It can be heavily tuned to the problem at hand with various parameters \cite{Geurts2006}.

\subsubsection{AdaBoost Classifier}
The AdaBoost (AB) classifier is an ensemble method which combines multiple weak learners (estimators) \cite{Freund1995}.
The later weak learners in the chain are tweaked in order to solve those samples that were misclassified by the previous learners \cite{Freund1995}.
The result of each weak learner is combined into a weighted sum used to classify the sample \cite{Freund1995}.

\subsubsection{Gradient Boosting Classifier}
The Gradient Boosting (GB) classifier is a variant of the AdaBoost classifier.
If the exponential loss function is used, then it turns into the AdaBoost classifier.
The model creates multiple decision trees that are, in each iterative stage, fitted with the negative gradient of the multinomial deviance loss function.

\subsection{Exploratory Data Analysis}
Exploratory data analysis (EDA) is a broad concept containing various techniques \cite{Hoaglin2003, Tukey1977, Velleman1981} that can be used to explore and analyse data.
Some early techniques include stem-and-leaf plots and box plots.
EDA can be seen as applying tools and statistics in a meaningful way.
It could, for example, be used to perform variance analysis, detect outliers or smooth the data \cite{Hoaglin2003, Tukey1977, Velleman1981}.

\subsubsection{Word clouds}
In the field of NLP, the use of word clouds can help visualise important aspects of the data.
Word clouds are generated by simply counting the occurrences of each word in the data set and presenting the most common words in an image.
The most frequent words are typically presented with a larger font.

\subsection{Evaluation Metrics} \label{sec:evaluation-metrics}
Common evaluation metrics in the NLP field are Accuracy (Equation \ref{eq:accuracy}), Precision (Equation \ref{eq:precision}), Recall (Equation \ref{eq:recall}) and F1 Score (Equation \ref{eq:f1-score}).
They are computed by calculating the number of True Positives (TP), True Negatives (TN), False Positives (FP) and False Negatives (FN).

\begin{equation} \label{eq:accuracy}
    Accuracy = \frac{TP + TN}{TP + TN + FP + FN}
\end{equation}

\begin{equation} \label{eq:precision}
    Precision = \frac{TP}{TP + FP}
\end{equation}

\begin{equation} \label{eq:recall}
    Recall = \frac{TP}{TP + FN}
\end{equation}

\begin{equation} \label{eq:f1-score}
    F1\ Score = \frac{2 * Recall * Precision}{Recall + Precision}
\end{equation}

