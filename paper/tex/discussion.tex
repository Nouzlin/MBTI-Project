\section{Discussion} \label{sec:discussion}

This chapter will discuss the limitations of the data set and the different results obtained for the three filter level models. 

\subsection{Data Set Limitations}

The data set used in this work contains numerous limitations, which are covered in the sections below.

\subsubsection{Self-Referencing Data}
The MBTI of a user is frequently self-referenced in the posts by that user. 
The self-referencing means that the unfiltered models learn that it is best to predict the most frequent MBTI in the document.
The Extra Trees model achieves an accuracy of 64.9\% with the unfiltered data set.
The accuracy could potentially be higher if one would create a simple if-else case and predict the MBTI that is the most frequent in the document.
However, self-referencing is not a realistic model to use.
It is a symptom of the origin of the data set; the forum the data originates from is used to discuss Myers-Briggs types.

\subsubsection{Myers-Briggs Type Indicator Imbalance}
The data set is overly represented by certain MBTIs, see Figure \ref{fig:types-barplot} in Appendix \ref{appendix:types-barplot}.
The INFP type is represented the most in the data set (21\%).
Always predicting the majority type would lead to a train and test accuracy of 21.1\%.
However, the precision, recall and F1 score metrics would be 0.04, 0.21 and 0.07, respectively.
Thus, the results obtained and presented in Table \ref{tab:filter-results} and in Table \ref{tab:extreme-results} suggest better performance than the null hypothesis classification would.

The validity of MBTI, based on the writing of a user, can not be confirmed by this work.
A weak connection is suggested by the models trained, but further work is needed, reasonably on a different data set.
The best classifier trained is slightly worse than the 37\% accuracy achieved by Ma et al. in \cite{maneural}.

\subsubsection{Limited Post Length}
The EDA analysis performed indicates that some posts by users are cut-off.
A post can end mid-sentence and simply be followed by "$\ldots$".
This limits the amount of information that could be extracted from the posts, as the length of posts could be used as a feature.

\subsubsection{Tagging}
Only URLs were tagged in this work.
It would perhaps have yielded better results if also numbers and user-mentions (prefixed by "@") would have been tagged or removed.

\subsubsection{Filter Levels and Feature Types}
The results improved with the number of topics used for the type-filtered models.
The improvement could be explained by a larger corpus used when creating the LDA models.
When extreme terms are filtered the number of topics needed seem to decrease.

The differences between the two feature types are minimal.
The results indicate that the topic distribution (TD) features perform as well as the topic terms (TT) features.
One reason for the similar performance could be the feature reduction happening in both cases.
The topic distribution feature is bound to the number of topics used in the LDA.
The topic terms feature is bound by the number of topics used, and the number of top terms extracted for each topic.
The results could have potentially differed if a larger or smaller number of top terms were extracted.

\subsubsection{Corpus}
The corpus is created from the training data, which means that terms not present in the training data set are ignored.
The ignoring of unknown words in new documents worsens the generalisation capabilities of the models.
The limited data set further supports this claim.
A larger data set, combined with textual data from different sources, would probably yield better results.


