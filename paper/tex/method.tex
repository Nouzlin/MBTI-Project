\section{Method}

This section covers the data set used in this work and the preprocessing and feature extraction used.
The implementations of the classification models used are briefly mentioned.
The work was implemented using Python\footnote{https://www.python.org/} version 3.6.2.
\subsection{Data Set}
The data set used in this work was downloaded from Kaggle\footnote{https://www.kaggle.com/}.
It contains data from 8675 Personality Cafe \footnote{http://personalitycafe.com/forum/} forum users.
Each data entry contains the 50 latest posts from a user, together with the MBTI of the user.
30\% of the data was used as a test set and the training set consisted of the remaining 70\%.
\subsection{Exploratory Data Analysis}
A simple EDA was performed to better understand the data.
This work mainly used word clouds, bar plots and document content analysis.

\subsubsection{Word clouds}
The Python library \texttt{word\_cloud}\footnote{https://github.com/amueller/word\_cloud} was used to generate word clouds.
First a word cloud was generated from all the documents in the data set.
The idea behind this general word cloud was to see if there were any terms or topics that needed further investigation.
After the general word cloud was generated the documents were partitioned on their respective MBTI labels.
A word cloud was then generated for each MBTI label.
Appendix \ref{sec:wordclouds} contains all the word clouds generated.

\subsection{Preprocessing} \label{sec:method-preprocessing} 
The data set was preprocessed according to the following steps:

\subsubsection{Standardising}
The data set was standardised so that any leading or trailing "\texttt{'}" characters were removed.

\subsubsection{Tagging}
All URLs in the data set were replaced with a \texttt{<URL>} tag.
Only URLs were replaced with a tag in this work.

\subsubsection{Tokenising}
The data was tokenised using the tokeniser from the \texttt{NLTK}\footnote{http://www.nltk.org/} package.
The \texttt{NLTK} package also contains an English stop word list that were used to remove stop words.

\subsubsection{Stemming}
The Snowball stemmer from the \texttt{NLTK} package was used to stem the tokens in the preprocessing step. 

\subsection{Features}

The features covered in Section \ref{sec:theory-features} were extracted and tested with the models in Section \ref{sec:theory-models}.
This section covers the implementation-specific details for the feature extraction used in this work.

\subsubsection{Filter Levels} \label{sec:method-filter-levels}
Three \emph{filter levels} were introduced in order to try and improve the quality of the different feature models:

\paragraph{Normal Filter} \label{sec:method-normal-filter}
This filter level only applies the first two steps in the preprocessing mentioned in Section \ref{sec:method-preprocessing}.

\paragraph{Type Filter} \label{sec:method-type-filter}
This filter level applies the standard preprocessing, but it also removes any MBTIs found in the data set.
The filtering is done by adding the MBTIs to the stop-word list.

\paragraph{Extreme Filter} \label{sec:method-extreme-filter}
This filter level applies both the standard preprocessing and the removal of MBTIs, as in the case of the Type Filter.
It also removes any tokens that occur in less than five documents or in more than 50\% of the documents.

\subsubsection{Latent Dirichlet Allocation}
The library \texttt{gensim}\footnote{https://radimrehurek.com/gensim/} was used to implement the LDA model.
Six LDA models with 10, 16, 25, 50, 75 and 100 topics were trained for each filter level.
The LDA models were used to extract the topic terms and topic distribution feature vectors covered in Section \ref{sec:theory-features}.


\subsubsection{Topic Terms Feature Vector}
The topic term feature vector covered in Section \ref{sec:theory-topic-terms} was implemented using 10 topic terms.
The feature vector was normalised for the LinSVC, GB and ET classifiers.


\subsection{Model Implementation and Training}
The models used the implementations in the \texttt{scikit-learn}\footnote{http://scikit-learn.org/stable/} package for Python.
The main parameters for each model was tuned using the \texttt{GridSearchCV} class, which allows a grid of parameter values to be tested for each model using cross-validation. 
The cross-validation used in this work was \texttt{StratifiedKFold} using five folds.
The stratified $k$-folds tries to maintain the imbalance of the data set in the different folds. 
The \texttt{F1\_micro} score was used to choose the best cross-validated model during the \texttt{GridSearchCV} parameter tuning.
The micro score globally counts all TPs, TNs, FPs and FNs, which is more suitable if the data set is imbalanced. 

\subsubsection{Initial Models} \label{sec:method-initial-models}
The first part was to use only the standardised and tagged data set (see Section \ref{sec:method-normal-filter}) in a LR model.
The classifier was tested using the bag-of-words model, the bigram model and the TF-IDF model.
The best LR model obtained through the \texttt{GridSearchCV} process was then evaluated on the test set.

The next step was to try the ET classifier on the standardised and tagged data set.
The features were extracted by using TF-IDF with a unigram model, followed by truncated Singular Value Decomposition (SVD) to reduce the number of features.
\texttt{GridSearchCV} was used to optimise the parameters for TF-IDF, truncated SVD and the ET classifier.

\subsubsection{Type Filter Models} \label{sec:method-type-filter-models}
The second part was to filter out the MBTIs from the data set, see Section \ref{sec:method-type-filter}.
Instead of using the built-in tokenisation methods in \texttt{scikit-learn}, a custom corpus was built using the \texttt{NLTK} library.
All the models briefly covered in Section \ref{sec:theory-models} were used on the type-filtered data set.
The parameters for each model was individually optimised using \texttt{GridSearchCV}.

\subsubsection{Extreme Filter Models}  \label{sec:method-extreme-filter-models}
The final part was to also filter out extreme words from the data set, see Section \ref{sec:method-extreme-filter}.
The same models are used as in Section \ref{sec:method-type-filter-models}.
The parameters were again optimised using \texttt{GridSearchCV}.